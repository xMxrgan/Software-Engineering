\section{Introduzione}

\subsection{Scopo del progetto}
Questo progetto ha come obiettivo lo sviluppo di un \textbf{sistema informatico di telemedicina finalizzato alla gestione clinica di pazienti affetti da diabete di tipo 2}. Il sistema nasce con l’intento di fornire uno strumento digitale che favorisca il monitoraggio continuo della malattia, semplificando la comunicazione tra medico e paziente e migliorando la qualità dell’assistenza.

Il sistema permetterà ai pazienti di registrare quotidianamente i propri dati clinici, mentre i medici potranno analizzare questi dati per valutare l’andamento della patologia, aggiornare le terapie e intervenire tempestivamente in caso di anomalie.

\subsection{Contesto clinico e motivazioni}
Il diabete mellito di tipo 2 è una \textbf{malattia metabolica cronica} che si manifesta con livelli elevati di glucosio nel sangue (iperglicemia), a causa di una produzione inadeguata di insulina da parte del pancreas o di una ridotta sensibilità delle cellule all’insulina stessa. Si tratta di una patologia altamente diffusa: in Italia coinvolge oltre 3,5 milioni di persone, e rappresenta più del 90\% dei casi di diabete diagnosticati, come riportato dalla Società Italiana di Diabetologia.

A livello globale, la prevalenza del diabete di tipo 2 è in costante aumento, in particolare nei paesi occidentali, dove fattori come l’invecchiamento della popolazione, lo stile di vita sedentario e le abitudini alimentari scorrette giocano un ruolo determinante. Il diabete non trattato o mal gestito può portare a complicanze gravi, tra cui problemi cardiovascolari, neuropatie, retinopatie e danni renali.

Per questo motivo, è essenziale garantire ai pazienti strumenti che li aiutino a mantenere un controllo regolare della patologia e permettano ai medici di intervenire con precisione, anche a distanza. Un sistema di telemedicina risponde esattamente a questa esigenza, combinando la raccolta strutturata dei dati clinici con funzioni di analisi, visualizzazione e allerta.

\subsection{Obiettivi del sistema}
Il sistema progettato si propone di soddisfare diverse esigenze pratiche e cliniche attraverso le seguenti funzionalità chiave:
\begin{itemize}
    \item \textbf{Registrazione dei dati clinici da parte del paziente}: ogni utente potrà autenticarsi ed effettuare l’inserimento dei valori della glicemia rilevati prima e dopo i pasti, registrare eventuali sintomi (come spossatezza, nausea, mal di testa) e segnalare l’assunzione di farmaci o insulina, specificando il tipo, l’orario e la quantità assunta.
    \item \textbf{Monitoraggio da parte del diabetologo}: i medici avranno accesso ai dati inseriti dai pazienti, potranno prescrivere terapie personalizzate (includendo dettagli su farmaci, dosaggio, frequenza e modalità di somministrazione) e aggiornare le informazioni cliniche del paziente. Potranno anche consultare l’andamento della glicemia su base settimanale o mensile, attraverso visualizzazioni sintetiche.
    \item \textbf{Sistema di alert}: verranno generati avvisi automatici in caso di valori glicemici fuori soglia, di mancata assunzione dei farmaci per più di tre giorni consecutivi o di altri comportamenti a rischio. Gli alert saranno indirizzati sia al paziente (per ricordargli di seguire correttamente la terapia) sia al medico (per valutare eventuali modifiche al trattamento).
    \item \textbf{Tracciabilità e responsabilità}: ogni azione effettuata da un medico sarà registrata, in modo da garantire trasparenza e attribuzione delle responsabilità nel processo clinico.
\end{itemize}
In sintesi, il sistema punta a supportare la gestione del diabete in modo moderno, efficiente e personalizzato, promuovendo un maggiore coinvolgimento del paziente nel proprio percorso terapeutico e facilitando il lavoro dei medici attraverso strumenti digitali integrati.